\documentclass[10pt]{article}
\usepackage{times}
\usepackage{latexsym}
\usepackage{graphicx}
\usepackage{amsmath, amsthm, amssymb}

% Use more of the page
\setlength{\topmargin}{0.0in}
\setlength{\textheight}{9.0in}

% Gives equations room in which to expand on the right
\addtolength{\oddsidemargin}{-1.0in}

% Various shorthands for mathmode
\newcommand{\bs}[1]{\boldsymbol{#1}}
\newcommand{\mc}[1]{\mathcal{#1}}

\newcommand{\argmax}[1]{\underset{#1}{\operatorname{argmax}\:}}
\newcommand{\argmin}[1]{\underset{#1}{\operatorname{argmin}\:}}
\newcommand{\mmax}[1]{\underset{#1}{\max\:}}
\newcommand{\mmin}[1]{\underset{#1}{\min\:}}

\newcommand{\norm}[1]{\mathcal{N}\left( #1 \right)}
\newcommand{\pd}[2]{\frac{\partial #1}{\partial #2}}
\newcommand{\logp}[1]{\log\left( #1 \right)}
\newcommand{\lnp}[1]{\ln\left( #1 \right)}
\newcommand{\expp}[1]{\exp\left( #1 \right)}
\newcommand{\abs}[1]{\left| #1 \right|}
\DeclareMathOperator{\sgn}{sgn}

% Ordinal superscripts; these should work in normal or mathmode
\newcommand{\superscript}[1]{\ensuremath{^{\textrm{#1}}} }
\newcommand{\subscript}[1]{\ensuremath{_{\textrm{#1}}} }
\newcommand{\xth}[0]{\superscript{th}}
\newcommand{\xst}[0]{\superscript{st}}
\newcommand{\xnd}[0]{\superscript{nd}}
\newcommand{\xrd}[0]{\superscript{rd}}

\begin{document}

\title{Class: Name} 

\author{Brian Romanowski\\
btr@msu.edu} 
% \date{April 15, 2010}

\maketitle

%%%%%%%%%%%%%%%%%%%%%%%%%%%%%%%%%%%%%%%%%%%%%%
% Code fragments
%%%%%%%%%%%%%%%%%%%%%%%%%%%%%%%%%%%%%%%%%%%%%%

% \begin{figure}[h!]
% \centering
% \includegraphics[scale=0.60]{lassoWeights.loose.noOffset.eps}
% \caption{Weights $\beta$ for lasso regression as $\lambda$ varies.}
% \label{lassoWeights}
% \end{figure}

% \begin{table}[h!]
% \begin{center}
% \begin{tabular}{| l || c | c | c | }
% \hline
% $\lambda$ & training & validating & testing \\ \hline
% \end{tabular}
% \caption{Logistic regression accuracy on the heartstatlog corpus, averaged accuracy across 5 folds of the training data}
% \label{p1results}
% \end{center}
% \end{table}

\section*{Question 1}

\end{document}
