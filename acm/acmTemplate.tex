% THIS IS SIGPROC-SP.TEX - VERSION 3.1
% WORKS WITH V3.2SP OF ACM_PROC_ARTICLE-SP.CLS
% APRIL 2009
%
% For tracking purposes - this is V3.1SP - APRIL 2009

\documentclass{acm_proc_article-sp}

% \usepackage{latexsym}
% \usepackage{graphicx}
% \usepackage{amsmath, amsthm, amssymb}
\usepackage{hyperref}

\newcommand{\argmax}[1]{\underset{#1}{\operatorname{argmax}\:}}
\newcommand{\argmin}[1]{\underset{#1}{\operatorname{argmin}\:}}

\begin{document}

\title{Your Title Here}

\numberofauthors{1} 
\author{
\alignauthor
Brian Romanowski\\
%        \affaddr{Institute for Clarity in Documentation}\\
%        \affaddr{1932 Wallamaloo Lane}\\
%        \affaddr{Wallamaloo, New Zealand}\\
       \email{btr@msu.edu}
% 2nd. author
% \alignauthor
% G.K.M. Tobin\titlenote{The secretary disavows
% any knowledge of this author's actions.}\\
%        \affaddr{Institute for Clarity in Documentation}\\
%        \affaddr{P.O. Box 1212}\\
%        \affaddr{Dublin, Ohio 43017-6221}\\
%        \email{webmaster@marysville-ohio.com}
}

\date{7 April 2010}

\maketitle
\begin{abstract}
Chicken, chicken chicken.
\end{abstract}

% TODO: Find categories?
% A category with the (minimum) three required fields
% \category{H.4}{Information Systems Applications}{Miscellaneous}
%A category including the fourth, optional field follows...
% \category{D.2.8}{Software Engineering}{Metrics}[complexity measures, performance measures]

% TODO: Keywords?
% \terms{Theory}

% \keywords{ACM proceedings, \LaTeX, text tagging} % NOT required for Proceedings

\section{Introduction}
Chicken.

\section{Related Work}
Chicken.

\section{Dataset}
Chicken.

% \begin{figure}[ht!]
% \centering
% \includegraphics[scale=0.20]{TreasureScreenshot_DeskCpu2.photoshopBrighter.eps}
% \caption{Treasure domain screenshot. Visible entities include a computer, plant, chair, and desk.}
% \label{treasureScreenshot}
% \end{figure}

% NOTE: This figure goes across the two columns
% \begin{figure*}[ht!]
% \centering
% \includegraphics[scale=0.50]{TreasureEyeGazeSpeech.Qu2009.btrAddedFixs.eps}
% \caption{Speech and eye gaze fixation data in the Treasure dataset.  Notice that there are often multiple fixation candidates due to the proximity of entities and the resolution of the eye tracker.}
% \label{treasureSpeechAndEyeGazeAlign}
% \end{figure*}

\section{Evaluation}
The open source Sphinx 4 platform \cite{sphinx4} was used to recognize user speech from the Treasure corpus. \footnote{http://cmusphinx.sourceforge.net/sphinx4/}
The Sphinx 4 speech recognition tool was used with a pre-trained HUB4 acoustic model to recognize speech.
\footnote{The ``HUB4\_8gau\_13dCep\_16k\_40mel\_133Hz\_6855Hz.Model'' model was used.}

\section{Results}
The best performing systems and associated parameter values are shown in table~\ref{bestResults}.

\begin{table}[ht!]
\begin{center}
  \begin{tabular}{ | l | c | c | c | c | c | c | }
    \hline
    Metric & $\lambda$ & LM Weight & Result (\%) \\ \hline \hline
    Mean WER & 0.7 & 13 & {\bf 70.9} \\
                      & 1.0 & 7.5 & 71.7 \\ \hline
    Concept F1 & 0.3 & 11 & {\bf 64.3} \\
                      & 1.0 & 11 & 62.0 \\ \hline
    Mean WER (cropped) & 0.7 & 13 & 71.7 \\
                      & 1.0 & 13 & 71.7 \\ \hline
    Concept F1 (cropped) & 0.3 & 11 & {\bf 62.9} \\
                      & 1.0 & 11 & 61.8 \\ \hline
  \end{tabular}
  \caption{Parameter settings that maximize the various metrics (all shown as percentages)}
  \label{bestResults}
\end{center}
\end{table}

\section{Discussion}
Chicken.

\section{Future Work}
Chicken.

%ACKNOWLEDGMENTS are optional
% \section{Acknowledgments}

%
% The following two commands are all you need in the
% initial runs of your .tex file to
% produce the bibliography for the citations in your paper.

% \bibliographystyle{abbrv}
% \bibliography{sigproc}  % sigproc.bib is the name of the Bibliography in this case
\bibliographystyle{abbrv}
\bibliography{foo}

% You must have a proper ".bib" file
%  and remember to run:
% latex bibtex latex latex
% to resolve all references
%
% ACM needs 'a single self-contained file'!
%
%APPENDICES are optional
%\balancecolumns

% \appendix
% %Appendix A
% \section{Headings in Appendices}
% \subsection{References}
% Generated by bibtex from your ~.bib file.  Run latex,
% then bibtex, then latex twice (to resolve references)
% to create the ~.bbl file.  Insert that ~.bbl file into
% the .tex source file and comment out
% the command \texttt{{\char'134}thebibliography}.
% This next section command marks the start of
% Appendix B, and does not continue the present hierarchy

\balancecolumns
% That's all folks!
\end{document}
